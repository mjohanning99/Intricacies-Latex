\chapter*{A Dictionary of Hieroglyphs}
  \markboth{A Dictionary of Hieroglyphs}{A Dictionary of Hieroglyphs}
  \addcontentsline{toc}{chapter}{A Dictionary of Hieroglyphs}
A dictionary is one of the most important objects to use when studying a language and Egyptian is no exception. The most well-known one is the Wörterbuch der altägyptischen Sprache (Dictionary of the Ancient Egyptian Language) which was commissioned by the Prussian Academy of Sciences in 1897 and provided with a funding of M120,000\footnote{M stands for (Gold)mark, which was the official currency of the German Empire from the mid-19th century to the early 20th century.} by Emperor Wilhelm II. including nearly M40,000 from the Prussian Academy of Sciences itself, thus having a total funding of M160,000 which, at the time, was a very considerable amount of money\footnote{I was interested in finding out how much money this was exactly. Wikipedia states that, in 1913, US\$1 was equal to about M4.20. We can thus, using an inflation calculator, calculate that M160,000 must have been worth over US\$1,000,000 in today’s money (2019).}. In addition, it is also interesting to mention that work on this dictionary continued even through World War I and II and still has not halted completely. The majority of the ground-laying work was done by Adolf Erman — a very well-respected, German Egyptologist born in the mid-1800s — who has also written a number of grammatical works regarding the Egyptian language. The work is astronomical in size, consisting of nearly 16,000 expressions spread out over approximately ten volumes, making it the most comprehensive Egyptian dictionary to have ever been created (Erman and Grapow, chap.Vorwort); I have seen all volumes of this magnificent work myself and merely looking through them is fascinating. I recommend everyone who is interested in learning the Egyptian language to take a look at it. I am, however, unaware of an English version of this dictionary existing and thus, knowledge of German would be beneficial in understanding it.

There also exists an online version of the dictionary which includes all the physical volumes with some additions and that can be searched as easily as a modern, online dictionary called Thesaurus Linguae Aegyptiae (Dictionary of the Egyptian language).
